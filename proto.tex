\subsection{Point-To-Point Transmission}
The most essential part of WHCS is the ability for the modules and the base
station to be able to communicate wirelessly. In our research and prototyping
phase we made sure that this feat would be achievable. To ensure that we were
able to communicate using our radio transceivers we set up a prototype for
point{}-to{}-point transmission. The setup involved the use of two breadboards
each populated with a microcontroller and a radio transceiver. One
microcontroller out of the two was operating as the symbolic base station. The
HC{}-05 BlueTooth module was connected to the Atmega328 microcontroller and
through this module we were able to administer tests with the prototype setup.
The microcontroller acting as the base station had a routine that enabled
reading and writing to the NRF24L01's registers.  We were able to ensure that
the state of the radio transceiver is the state that we needed to communicate.
The other microcontroller was connected to the other radio transceiver as well
as a TTL{}-serial module for connecting to a computer{}'s terminal. The control
module microcontroller also had LED{}'s connected to two of the GPIO pins. The
setup that we created is shown in \autoref{fig:ptp-tx} This setup allowed for
us to make sure that the radio transceivers were able to send packets to one
another and that the packets could be read into the microcontrollers.

\ucfgfx[width=15.24cm,height=8.572cm]{fig:ptp-tx}{a81PointtoPointTransmissionexpected1pg-img001.png}{Point to Point Transmission Prototyping Setup}

When we finished connecting and hooking up this prototype setup
we were able to run the routine on the microcontroller that acted as the base
station to take input from the BlueTooth module. Using an Android phone we
communicated with the microcontroller and manipulated registers of the NRF24L01
to ensure that the microcontroller was communicating with the device using the
correct timing protocol. Once we ensured we were correctly interfacing with the
NRF24L01 from the base stations side we connected the NRF24L01 to the control
module chip. The control module chip was connected to a computer terminal via
the TTL{}-serial chip located in the top center of Fig. 8.1.1. We made sure
that we were able to interface with the radio transceiver for this
microcontroller in the same way as the first. Once both radio transceivers were
confirmed to be connected and interacting correctly we ordered the base station
to transmit data to the control module.  The command was initiating from the
Android application. The data was transmitted to the control module and was
successfully received. As a result, one of the LEDs attached to the control
module breadboard toggled to an on state.

This prototype was able to give us a good gauge of how feasible
our approach to designing a wireless home automation solution was. We were 
successfully able to get an Android device to communicate to a stripped down
base station and then to a stripped down control module. The actions that were
done with this prototype will be the core of WHCS. Every action done in the
system centers around the ability to communicate between microcontrollers and
to and from the mobile phone.  When the control module microcontroller is doing
more than just toggling an LED is when WHCS will be impressive.

\subsection{Rogers Board Etching Prototyping}
In our design we wanted to make the most use out of prototyping as possible.
the worst mistakes are always the mistakes that are discovered to be mistakes
in the last stage of a design. The first way to go about prototyping is to test
things at a very low level. Checking if individual subsystems work the way that
you would expect them to work, this level can definitely be done with a
breadboard. Once the individual subsystems have been tested the next step is to
check how these systems integrate, this too can be done with a bread board. At
this point you already have a complete proof of concept. The final stage in
prototyping is to do everything possible to make a design that is as close to
what the final product will be as possible. Initially we thought we were etching our own powerboards. We figured that testing the control
modules and base station with an etched PCB design would be the best way to
know if our drawings from our gerber files lined up the way we wanted them too.
As a prior step you can also compare dimensions of the parts of the design with
the gerber file drawing by simply printing out the gerber file on to copy paper
and placing elements on top. Etching would have added add an even further advantage
in that it will allow us to see if the electrical connections in our drawing
worked out the way we would expect them too. If we had proved the concept to
the level that it has been etched on to practice boards we would have known that our design
would work as it was intended to work. We did however decide not to make etched boards when prototyping our designs. We were pretty confident in our designs and thought it would be best to get our boards fabricated as soon as possible.

Etching is actually a fairly simple process. If we had etched we would have used toner paper to transfer the gerber file from the computer onto the board. Next
we would have utilized TRF paper to further protect the design. Finally we would have used an etchant to eat away the copper that is not part of our design. There
are many different options for etchants that could have been chosen in order to
complete this step. The most easily accessible though is made by combining
vinegar, hydrogen peroxide, and salt together. After using the etching solution
it is important to dispose of it properly (it can{}'t just be poured down the
sink). Since the solution is very acidic balancing it with a strong base such
as baking soda or soap will prepare the solution for a better disposal. We would have used rogers corporation boards because they provide free
laminate samples to students. Once the board was etched we would have simply soldered on
the components to the board. 

\subsection{WHCS Proto-Panel}
\label{sec:proto-panel-proto}
Presentation plays a huge part on how the public feels about a product. This is
why it was so important that the display of the project was well put together.
Not only is it important to have a nice display for the purpose of it being a
proper representation of our design, but also so that it is aesthetically
appealing. If WHCS were to be launched into industry, marketing would play a
huge part in it{}'s success. People{}'s first impressions are always driven by
what they see. If what they see causes them to believe that the product is of
high quality, they are less likely to be highly skeptical of how the product
performs. In this section we will be going into detail about how we made our display
in order to showcase the functions of WHCS. This section will not go into the
practical side of how the display is coming together; rather it will lay the framework for what goals we want to
accomplish with the display.

Our design is not a plug and play design, therefore installation will require more than simply plugging in the system. We had to find some way to
duplicate what the installation of a house would look like. In a home what
we{}'re provided with is interior wiring. When we are actually presenting our
idea what we{}'ll be presented with is an outlet that we can use to draw power
from. Therefore the first thing we did was convert this outlet
back into wiring. To do this we took a basic power cord that had a hot, neutral, and ground wire and spliced it with 14 gauge wire that we got from home depot that is meant to be used in a home. We spliced this wire into may different branches and used it to power the control modules and the base station. 

We wanted to make the display as accurate of a representation of
what would be found in a home as possible. Therefore we tried to follow as
many codes and standards for home construction as possible. Since we ourselves
do not have a homeowners electrical permit we are not equipped to actually
install the system in a real home. Yet for demonstrative purposes we did
fine to follow codes and present them in our display. In \autoref{sec:proto-panel}
we go into further detail of what was done to follow these codes and
standards.

Our display consists of the frame, the wiring, and drywall.
Each made to follow standards. We also decided to make it such that each board was displayed in our proto-panel. To do this we replaced part of the drywall with plexiglass and displayed the temperature sensor control module and the power board.   

\subsubsection{Materials}
The first thing we needed was a plug. It was important that we used a three prong
plug because wiring in the home uses three wires. In addition to the wiring, the
interior walls of homes consist of drywall, insulation, and a wooden frame. The
wooden frame of homes is made out of 2 by 4 wood. These pieces of 2 by 4 wood
are usually put together with either screws or nails. Drywall will also be
needed to provide the presentation side of our wall. For drywall all we
needed was enough drywall to cover the entire wall along with some screws
in order to attach the drywall. Insulation wasn't necessary since we
weren{}'t worried about temperature or sound insulation for our project. Also
insulation in a regular home wouldn{}'t interfere with the installation of our
project so really the insulation is irrelevant.

In addition to these basic materials made to construct the walls
we needed a few other things. First we needed the actual control module
and base station boards. If this design were a final product these boards along with the power board would be housed
in a case that would be attached to the wooden framework. However for our project we simply mounted the boards directly to the frame of our display. For the light control
module we needed to buy a wall mounted lamp with a three wire connection. For the
outlet we needed to buy an outlet. For the outlet and light
control modules the relays were  placed along the hot wires in order to
switch them on and back off. To display the outlet control module we needed
to have something plugged into our display board outlet (we used a
coffee maker). Additionally we needed a door knob,
the strike, a lamp shade, and some plexiglass to display a power board and the temperature sensor control module.


\subsubsection{Dimensions} The dimension of our project was pretty arbitrary,
it just had to be large enough to fit each control module along with the
base station. The first thing that we decided was whether or not
we wanted to use a full size door for our design. Although the idea was
tempting, because it would really give the user the feel of a home experience,
we decided against it mostly because of weight. If the fame had to be of that
size all the wood used in the frame along with the weight of an actual door
would have made the project very difficult to move around. Also it would have been extremely bulky. Getting an entire door (actually even larger
than an entire door because of the other attached components) through a door
can be quite a struggle, add weight to the mess and you{}'re asking for
difficulties.

The door we used was a homemade door. Fortunately
because we designed the frame that will be used for the wall, we
simply made the gap in-between the studs the same size as the door we wished to
make. We decided to use a 2 by 12 piece of wood in order to make the door. As a rough estimate we decided that 2 feet by 2 feet would be a
large enough area of space to display each individual module. The total square footage of
the five control modules would be 20 square feet. We decided that a 4 by 5 feet
display wall would showcase our design quite nicely. Yet ended up going with a 4 by 6 feet display because it was easier to view and play with.

\subsubsection{Sketch} This section shows the original mock up of our 4 by 5 display (even though we ended up using a 4 by 6 display). The most important interactive parts of our design are the door access
control module and the base station. We made sure that these two modules
were at an acceptable height where they could be interacted with. In the mock up we decided to
place three of the modules on the bottom portion and two modules on the top
portion. The two on top were at a more accessible height therefore the door access
and the base station were placed there for our mock up. The light outlet and temperature
sensor were placed on the bottom half. To make things look symmetrical the
lamp was placed in the middle while the outlet and the temperature sensor
were placed on the side. The design we came up with in the mock up is shown below in  \autoref{fig:proto-sketch}.

\ucfgfx[scale=0.45]{fig:proto-sketch}{a83expected2pagesprotopanel-img001.png}{Mock up design of WHCS display board}

What we actually ended up building is shown in \autoref{fig:proto-panel} below.

\ucfgfx[scale=0.25]{fig:proto-panel}{11700729_10205701511914436_783349522353423227_o.png}{Built WHCS display board}