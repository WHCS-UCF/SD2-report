This section contains an overview of the components necessary to create the
Wireless Home Control system. Without going too deep into implementation
details, an abstract view of WHCS is provided. Block diagrams are utilized to
show the interaction between WHCS subsystems. The information in this section
covers the central components necessary to realize the system and fulfill our
objectives.

% Section 4.1
\subsection{Base Station}
The base station of WHCS is the central component. It is the only component
that is involved in everything that is done with WHCS. Any command going
to an endpoint passes through the base station and any information coming
from an endpoint goes through the base station. The base station acts as the
smart middleman between the user and the control modules that they are
targeting. The base station is a small portable device that can plug into a
home outlet for power. The user needs to be able to interface with the base station in order to control the
system. To facilitate this interaction there is a mobile WHCS application
that can be used to communicate to the base station, and there is also a
touch enabled LCD. Both of these methods are available to control the base
station. The base station is designed to be the shape of a rectangular prism and has an LCD flush with the surface of the top of the prism. This way the PCB
is hidden away and does not have to be known to the user. This provides
an aesthetically pleasing look to the base station and hides away its inner
workings. \autoref{fig:bs-ext-design} shows what the base station would look like
including the LCD, the inner PCB, and the plug for power in a home installation. The size of the base
station is based off of the size required for the printed circuit board.

\ucfgfx[width=7.27cm,height=6.443cm]{fig:bs-ext-design}{a5Expected4pages-img001.png}{Base Station Exterior Design for WHCS}

{\color{black} The base station requires a special printed circuit board
that is different from the control modules. It is the only board in the
system that has to support BlueTooth communication. This is how the base
station communicates with the Android phone. It also needs to have
space for the installation of the LCD screen. The LCD screen is able to be
mounted on the base station PCB so it can be pushed through the display box.
Using this architecture the LCD is all the user has to see. Like the
control modules the base station has a radio transceiver chip. This is the part of the base station that allows it to communicate to the control
modules who have a similar radio transceiver. The LCD, BlueTooth chip, and
radio transceiver connect to the microcontroller that is chosen
for the base station. This whole system is powered by a 120V AC step down
circuit. The base station also has room for a
programmer pinout that way any errors in design can be fixed in circuit. The
block diagram of the components for the base station is depicted in
\autoref{fig:bs-pcb-block-diagram}.}

\ucfgfx[width=15.24cm,height=11.43cm]{fig:bs-pcb-block-diagram}{a5Expected4pages-img002.png}{Base Station PCB Block Diagram}

{\color{black} As shown in the block diagram the programmer pinout and the
radio transceiver will both be connected using the SPI bus of the
microcontroller. SPI allows for selecting which chip to utilize at any given
time so therefore the bus sharing conflict will be able to be resolved.}

% Section 4.2
\subsection{Control Module}
The control modules of WHCS are the boards that connect to the target
endpoints around the home. Control modules can be of different types due to the
fact there are multiple targets in the system. There are light/outlet
modules, door modules, and sensor modules. All of these different types of
control modules have similarities that can be taken advantage of in order to
create a single control module design that can be adapted to the specific
endpoint it is targeting. Every control module has a power supply
circuit, a microcontroller, and a radio transceiver at a minimum. The rest of
the circuitry revolves around what the control module is meant to interact
with. For example the circuit to activate the electronic strike on a door 
needs to switch the electronic strike power, while a temperature sensor circuit
needs to do analog to digital conversion. There are two options that these
possibilities brought about for the creation of control module boards. One option
was to create a control module design that has room to solder the components
necessary to interface with all of the supported endpoints of WHCS on one
board. This meant the control module would have the foundation for adding the
electronic strike circuit, the light/outlet circuit, and the temperature
circuit all at once. All that would need to be done is to connect the circuit
to the actual target of choice, and if desired the control module could be
removed and hooked up to a different type of target. The second option was to
create a printed circuit board design that has a section dedicated to the
implementation of a single control module endpoint circuit. This way when the
printed circuit board is assembled we could custom solder components based on
what that control module was being used for. Either of these options would have worked. The
option where all control module boards are able to support all three control
module modes is the more involved option so that is the option that we considered
for our design. \autoref{fig:ctrl-mod-blk-diagram} shows the block diagram for a control
module. The blocks labeled radio transceiver, power, and microcontroller are
the only ones necessary for every single control module. The three blocks on
the lower left are the circuits that interact with specific endpoints around
the home. These are the parts of the control module that can either all be
included in every control module, or can appear as lone circuits on every
control module.

\ucfgfx[width=15.24cm,height=11.43cm]{fig:ctrl-mod-blk-diagram}{a5Expected4pages-img003.png}{Control Module Block Diagram}

% Section 4.3
\subsection{BlueTooth Capable Phone}
{\color{black} The Wireless Home Control System was based around the capability
of interacting with the system from a mobile phone. A phone with BlueTooth
communication capabilities was required for the system to operate at its full
capabilities. With the selected BlueTooth phone platform, a software
application was developed to allow users to interact with WHCS. It is 
through this application that users can monitor the state of their home and
remotely interact with targeted appliances. The phone application features 
a low{}-complexity graphical user interface as well as the ability to use
speech{}-recognition activation. }

\subsection{Software}
WHCS contains a network consisting of its base station, its peripheral
control modules, and the user{}'s BlueTooth capable mobile phone. This created
the need for three distinct software models. These three different software
implementations follow a custom communication protocol in order to
transmit data throughout the network. Commands frequently flow from the
user{}'s mobile phone all the way to a control module that they want to
interact with.  The data flow can also occur in the reverse direction. It is
the goal of WHCS software to facilitate this interaction.  Phone to base
station communication is done through BlueTooth. The phone is able to
utilize BlueTooth libraries, while the base station microcontroller will
interface with a UART to control a BlueTooth module.  Communication between the
base station and control modules relies on a ported driver written for the
radio transceiver used in the system. The base station is the powerhouse
in terms of software design. It is at this central hub that the collection of
all active control modules is stored. The base station also
constantly has to be ready to accept requests from any control module or the
user. The main routine needs to have handlers ready for the LCD or the
mobile phone because either choice will be usable to interact with the system.
\autoref{fig:whcs-soft-block} shows the block diagram for the entire system{}'s
software. This is meant to provide insight into the operation of the large
components of the system

{\color{black} Apart from the networking aspect of WHCS the software also
needed to control the interaction with the targets of the individual control
modules. This is depicted in \autoref{fig:whcs-soft-block} as the bottom{}-most block.
This block had to be custom tailored based on the role of the control
module. The routines running on the microcontrollers need to know the
state of the appliance they are interacting with, and how to do any necessary
activation that is requested by the user. For sensor type control modules the
microcontroller needs to update at intervals in order to maintain correct
information. For the modules that are activating outlets/lights or controlling
the electronic strike the routine is as simple as toggling the state of a
solid state relay connecting to a GPIO pin. It is the communication
between subsystems that presented the biggest challenge for the microcontroller
based boards.}

{\color{black} The mobile application software has three main components,
the user interface, bluetooth communication handler, and speech recognition.
These blocks are shown at the top of \autoref{fig:whcs-soft-block} and together they
complete the mobile application. The user interface is the most involved
piece of software. The interface has to constantly display the state of
the system and provide intuitive use cases. All of the capabilities that we
created for WHCS had to be reflected by the user{}-interface that we
designed. The U.I. needed to be linked together with the BlueTooth
communication handler because the user expects state changes in
the system at the press of a button.}

\ucfgfx[width=15.24cm,height=11.43cm]{fig:whcs-soft-block}{a5Expected4pages-img004.png}{WHCS Software Block Diagram}
