\subsection{PCB House}
After carefully designing our schematics, we first need to decide on a PCB
manufacturer to use for our printed circuit boards. There are a few popular and
affordable options available for hobbyists and small run jobs. We weigh the
options available in the following sections.

\subsubsection{Seeed Studio}
This PCB house offers many services including a low cost PCB service. The
pricing model is based off of the number of layers, thickness, and size class.
The size class of 10cm x 10cm would have a base cost of around \$21.00. This is
quite fair and would meet our needs for manufacturing.  Seeed Studio is based
in China, which is something to take in to considering if we are ordering
boards and need them quickly. Also, even ordering initial boards would have a
long enough turn around time that we wouldn't be able to test and construct our
design iteratively. Seeed's cost is very competitive, but we would be losing
time to save cost.

\subsubsection{OSH Park}
OSH park is quite a popular option due to their low price point for small
boards. They are known by their distinctive purple boards. WHCS has decided to
choose OSH Park due to the fair price for a 2 layer board run. A single
run would yield 3 copies for \$5 a square inch. Unfortunately, this pricing
model will make any significantly large boards too expensive. For the control
modules, it is possible to create a very small design that still meets the WHCS
requirements.  WHCS has no special PCB requirements beyond being able to create
surface mount pads for the control module and base station boards.  The base
station PCB will be limited by the size of the LCD, but this is only around 2
by 3 inches.

Overall, OSH park is the best choice for WHCS' budget and technical
requirements. Its fast shipping time will ensure that we have plenty of time to test our board and make any additional runs if the worst happens. We ended up going with OSH park for all of our boards. We had two orders of three control modules, one order of base stations, and only one order of power boards to save on cost.

\subsection{Parts}
This section discusses how parts came into play of the manufacturing of
our PCB. With choosing parts there was much to consider. In addition to whether or not
the part performs as we needed them to, and whether or not these parts were
in stock we also had to take a look at how each part would be
implemented into our design.

\subsubsection{Footprint (SMD vs Through-Hole)}
The footprints of our board relied mostly on the parts that were chosen to
be part of our design. The parts needed for a surface mounted board and the
parts needed for a through hole boards are different. Therefore in order to
continue make our boards we had to make a design choice. In WHCS we considered
both through hole boards and surface mounted boards. Through hole board
technology is the older of the two technologies and is currently much less
popular than surface mounting. One of the of the advantages of surface mounting
is that it takes up less space allowing more real estate for parts for a given
board. Because surface mounting does not involve drilling it is simpler and
faster to construct. Although there are some advantages in through hole boards
for most applications surface mounted technology wins. Therefore in our design
we will be using surface mounted technology. With the exception of the power
board which will indeed take advantage of through hole for our heavier components. The footprints in our control module.

\subsection{Construction}
In this section discuss the assembly of our project. This
involves decisions that were made for the PCB board construction 
along with the construction of the display that shows how WHCS
operates.

\subsubsection{Soldering}
Once we received our etched boards we had to consider methods to solder on the
components. There are various methods that can be used to accomplish soldering.
Note that any of the methodologies described could have been used to accomplish the
task. The
first of the methods to be described is the use of hand soldering. This is by
far the simplest and the easiest conceptually for our group to implement. Every
member of our group has some experience soldering by hand. This method is cheap
as it only really requires a soldering iron, which most if not all members in
our group own.  The main take back with this method is that while it is easy by
methodology it is difficult in the fact that it can be time consuming.
Also since everything is done by hand this method tends to be a little messy
and may tend to look less professional. This is even
more so the case if we make use of flux, which can tend to make the board look
dirty. We initially wanted to use this method as a last resort.

\subsubsection{Reflow Oven}
We also considered refow. Reflow methods are different from hand soldering in that it separates the
placement of the components from the soldering. It tends to be less time
consuming and does a more consistent professional looking job. There are
various method that our group could use to reflow the solder for our PCBs. The
UCF amateur radio club has access to a professional reflow oven that we could have taken advantage of. Additionally, there are online methods describing how
to use a toaster oven to achieve the same affect.

In the end we actually did a combination of both reflow and hand soldering. We achieved the reflow by using solder paste and a heat gun. Hand soldering SMD parts was actually easier than expected with the use of magnifying glass.  

\subsubsection{Proto-Panel}
\label{sec:proto-panel}
In this section we will focus on some of the details of the construction of our
design, the general overview of how the display board was put together is
given in \autoref{sec:proto-panel-proto} This section will not explain how the
proto{}-panel came together, rather it will explain the non obvious
construction details. We discuss the design considerations that had to be
accounted for. This section takes into consideration the norms that go into
home construction. It also goes into the specifics of safety precautions as
well as regular sizes and spacing used in a home.

First let{}'s talk about the wiring. The amount of current that can be drawn
from an outlet is 15 amps if not 20 amps, therefore we had no problems in
drawing enough current from the outlet to power our board. 15 amps is more than
enough to satisfy our needs even when the current is divided into five different
applications. Something we had to consider was the gauge of the wire used. The gauge
of the wire dependents on the amount of current it can safely handle. As
discussed previously 1 amp is larger than anything we ever see from our
circuit. Therefore we designed our wire gauge for 1 amp.
However as stated before most homes are designed to be able to draw 15 if not
20 amps. This level of amperable is equivalent with wire of gauge 14 and 12. A
common brand of wiring used for these tyes of applications is Romex. Just for
the sake of being consistent with what is used in the home we used 14
gauge wire. To splice the wire we considered either hand soldering it or we using a wire
nut. Both are an acceptable methods for joining the wires, and both are used in
homes. Wire nuts are considered to be an easier/ faster method for doing the
job. While soldering is seen as the higher performing
link. We will made our links by soldering them because it is slightly more
professional than using a wire nut but really this is simply a matter of
preference.

When wiring something it is often a good idea to attach your wire at more
locations than simply the location where the connection is made. This way if
for whatever reason the wire is pulled the stress will not go completely to the
connection. Before splicing the wire and connecting the wires to the different
control modules and base station, it was a smart safety precaution to run
the unspliced wire through the wood framing and attaching it. After splitting the
wire it was a smart idea to continue the practice of attaching the
wire at more places than the connections. While doing our wiring we made sure to match the understood color scheme. In our design we have a single phase hot and
neutral and ground. In the US the ground wire matches with green, the black
wire matches with the hot wire, and the white wire matches with neutral.
\cite{link14}
Using these color codes made it easier to keep the project neat and
organized. It made it easier to avoid making mistakes in wiring our circuits.

Now let{}'s discuss the boards themselves and how they were placed into the
framework of our wall. originally we wanted to place the boards in some sort of casing that would easily be attached to the framework of the home. Our thinking was the less the number
of things needed for installation the better. The easiest way to do this would have been to attach them to or place them inside of the electrical boxes. Having only one thing to install per control module
or base station would have made real installation of our device more realistic for
actual use in the future. In our design we wanted to make use of electrical boxes
since they are used in home electrical wiring. The boxes can be either metal or
plastic, yet plastic boxes are a little easier to work with as they the holes
are easier to punch out. In the end we just mounted the boards to the frame, yet if we had more time we would have made some sort of casing similar to what was described. 
\cite{link15}

There are some specific considerations that were made with the different
control modules. First off for the light it was important that we used an actual
wall mounted light, as this is the type of light fixture that we would
be controlling in an actual home. It was important that it was not simply a plug in light,
because this would defeat the purpose of having light fixture control module. It would have been the same thing as the outlet module.
The wall mounted light we used came with three wires; a hot, a neutral, and a
ground wire. From these three wires we were able to install the fixture in
the same way as what would be expected in a real home. The hot wire is the wire
that we interrupted with the relay in order to switch the light on and
off.

The outlet we used was a GFIC. GFIC stands for ground fault
interrupter. Using this outlet provided an extra safety precaution. What a
GFIC outlet does is constantly compare the output current from the neutral wire
to the input current from the hot wire. If there{}'s a difference in current,
within the range of a few milliamps, the outlet will shut off in 20-30
milliseconds. In the case that someone were to be electrocuted by this outlet,
the current that goes into their body would cause a current leakage that would
cause the GFIC to have a current difference and thus shut off.  GFIC outlet are
normally required for kitchen and bathrooms. Since none of the members in our
group have extensive experience in working with AC power it was best that we
took every safety precaution available. We do not expect that homes that
actually implement our design will use GFIC outlets, it is simply an extra safety
precaution that our group decided to take. \cite{link16}


For the door control module we decided to make our own door with a 2'' by 12'' piece of wood. Since we did not buy the door but are custom made it ourselves
we needed to cut the holes ourselves. The first order of business is cutting the
door to length and leaving a frame of the right size. To cut the hole of the
latch we will needed to use a $\frac78$'' spade bit. The hole for the door knob
will had to be made with 2-$\frac18$'' diameter hole saw. A 1'' wide chisel was
used to cut out the recess of the latch. After all that cutting we were able to install the door
knob.
