In order to speed up development and promote compatibility the WHCS design
adheres to multiple standards. Using standards allows us to rely on certain
characteristics for the devices we use. Standards allow us to stand upon the
design decisions made by organizations before us.

\subsection{RS{}-232}
RS{}-232 (Recommended Standard) is a serial communication standard that
specifies hardware and software implementation for serial communication between
two connected devices. Our microcontrollers will be using a UART for debugging
and communication through BlueTooth. The UART module present in our
microcontrollers is based off of RS{}-232. The microcontrollers use logic
voltage levels which are are not the same as those specified in the standard.
This version of RS{}-232 is often referred to as TTL{}-serial (Transistor to
Transistor Logic). The UART also only uses the Rx and Tx lines specified in the
standard. The software and algorithm used to communicate with the UART is the
same as RS{}-232, so the deviation from the standard is hardware only.

\subsection{BlueTooth}
BlueTooth is the standard that we will rely upon for
communication between the base station of WHCS and the mobile application.
BlueTooth was originally standardized by IEEE as standard 802.15.1. The
standard is now maintained by the BlueTooth special interest group. Devices
that implement BlueTooth are able to connect to one another wirelessly to
exchange data serially. The wireless signal{}'s frequency is regulated to be
between 2400 and 2483.5 MHz. The base station will contain a BlueTooth module
which implements the BlueTooth standard. This module will allow communication
with the mobile phone, who{}'s hardware will support the BlueTooth standard.
BlueTooth{}'s serial communication characteristics make it well{}-suited for
communicating with a microcontroller{}'s UART. The UART to BlueTooth module
connection is what we will be performing.

\subsection{SPI}
SPI (Serial Peripheral Interface) communication is a de facto
standard so there is no regulatory body that develops or maintains the
specification. SPI is used in microchips for communicating to multiple
connected devices through one bus.  SPI features a slave select line that
allows the microchip (the master) to pick a target device (slave) to speak to.
WHCS will use SPI for interfacing with the radio transceiver module from the
microcontroller on the base station and control modules. SPI will also be used
to program the microcontrollers we use. The slave select feature will allow us
to have our radio transceiver still attached to the SPI bus while we are
programming our microcontrollers.

\subsection{FR{}-4}
FR{}-4 (Flame Resistant) is a standard for flame resistant
glass{}-reinforced epoxy laminate sheets. These provide the basis for printed
circuit boards. The FR{}-4 standard was developed for complying with the
regulations for flammable plastics set in standard UL94V{}-0. The core of PCBs
are built around a laminate sheet that meets the FR{}-4 standard.  We will be
populating printed circuit boards for WHCS so we will rely on this standard for
the proper operation and safety of our PCBs.

\subsection{Android Development Guidelines}
The Android developer{}'s guide has a list of guidelines that
are strongly recommended for Android applications. These are standardized
guidelines, but they are not standards because they are not required. These
guidelines are relevant to our project because we will be developing an Android
application and we will be adhering to the specifications. The guidelines all
have individual ID codes. Some noteable guidelines that we will be following
during the development of the WHCS application are UX{}-B1, UX{}-N1, and
FN{}-S1. Respectively these standards involve not redefining the functionality
of on system icons, supporting back button operation in applications, and not
leaving services such as BlueTooth open while an application is in the
background.

\subsection{ANSI/NEMA 1-15P, 5-15P, C84}
WHCS will get its source power from household mains. The
connectors for household mains, and the voltage levels provided are
standardized by ANSI/NEMA. The plugs that will allow the base station to be
connected to a United States structure{}'s outlets are standardized by 1-15p (2
prong) and 5-15P (3 prong). We can count on the input voltage of WHCS being
120v AC because of standard C84 which standardizes the power supplied to
household mains in the United States.
